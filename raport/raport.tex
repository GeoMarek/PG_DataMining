\documentclass[12pt, oneside, openany]{article}
\usepackage[utf8]{inputenc}
\usepackage{polski}
\renewcommand*{\figurename}{Rys.}
\usepackage{graphicx}
\usepackage{float}
\usepackage{geometry}
\geometry{
	left=25mm,
	right=25mm,%
	bindingoffset=10mm, 
	top=25mm, 
	bottom=25mm}
\title{
	Eksploracyjna analiza danych \\
	Światowy program szczepień przeciwko COVID-19
}
\author{
	Marek Grudkowski 156587
	\\
	Kamil Kaczmarkiewicz 171701
}

\begin{document}

\maketitle

\section{Ogólny opis danych}

Zbiór danych dotyczy aktualnego postępu poszczególnych krajów w szczepieniach przeciwko COVID-19. W dobie pandemii jest to niezwykle gorący temat. Naszym zdaniem warto się na nim skupić, gdyż może zawierać wiele ukrytych informacji, które mogą przydać się w walce z pandemią i przyspieszyć sam proces szczepień. 

\section{Cel eksploracji i kryteria sukcesu}

\section{Charakterystyka zbioru danych}
Zbiór danych zawiera ponad 13300 przykładów, które pochodzą z wielu różnych źródeł. Zazwyczaj są nimi organy krajowe lub lokalne, czy międzynarodowe organizacje. Dla każdego przykładu podane jest źródło i jego adres internetowy, co daje możliwość weryfikacji w przypadku jakichkolwiek wątpliwości co do poprawności danych. Dane zapisane są w jednym pliku w formacie csv i podzielone są na następujące kolumny:
\begin{itemize}
\item \textbf{country} - kraj, dla którego podawane są informacje o szczepieniu, atrybut nominalny w postaci ciągu znaków
\item \textbf{iso\_code} - kod ISO dla danego kraju, atrybut nominalny w postaci ciągu znaków
\item \textbf{date} - data wprowadzenia danych, atrybut nominalny opisujący datę
\item \textbf{total\_vaccinations} - bezwzględna liczba wszystkich szczepień ochronnych w danym kraju, atrybut numeryczny (liczba naturalna)
\item \textbf{people\_vaccinated} - liczba osób która otrzymała szczepionkę (przy dwóch dawkach liczona jest $\times2$), atrybut numeryczny (liczba naturalna)
\item \textbf{people\_fully\_vaccinated} -  liczba osób, które otrzymały cały zestaw szczepień, atrybut numeryczny (liczba naturalna)
\item \textbf{daily\_vaccinations\_raw} - dla danej pozycji liczba szczepień dla tej daty/kraju, atrybut numeryczny (liczba naturalna)
\item \textbf{daily\_vaccinations} - dla danej pozycji liczba szczepień dla tej daty/kraju, atrybut numeryczny (liczba naturalna)
\item \textbf{total\_vaccinations\_per\_hundred} - stosunek liczby szczepień do całkowitej liczby ludności danego dnia w kraju, atrybut numeryczny wyrażany w procentach
\item \textbf{people\_vaccinated\_per\_hundred} - stosunek liczby osób zaszczepionych do całkowitej liczby ludności danego dnia w kraju, atrybut numeryczny wyrażany w procentach
\item \textbf{people\_fully\_vaccinated\_per\_hundred} - stosunek liczby osób uodpornionych do całkowitej liczby ludności danego dnia w kraju, atrybut numeryczny wyrażany w procentach
\item \textbf{daily\_vaccinations\_per\_million} - stosunek między liczbą szczepień a całkowitą liczbą ludności na bieżący dzień w kraju, dodania liczba rzeczywista
\item \textbf{vaccines} - rodzaje szczepionek wykorzystanych w danym kraju, atrybut nominalny, ciągi znaków rozdzielone ukośnikiem
\item \textbf{source\_name} - źródło informacji, atrybut nominalny, ciąg znaków
\item \textbf{source\_website} - strona internetowa źródła informacji, atrybut nominalny, ciąg znaków
\end{itemize}


\section{Wyniki eksploracyjnej analizy danych}
rozkłady wartości atrybutów, korelacje pomiędzy wartościami atrybutów
wstępne ustalenia dotyczące zawartości zbioru 
\section{Uwagi dotyczące jakości danych}
dane brakujące, punkty oddalone, dane niespójne, dane niezrozumiałe,
\section{Opis wyników eksploracji}

w odniesieniu do celów eksploracji, czy dane są wystarczające, ewentualna rewizja celów

\end{document}
